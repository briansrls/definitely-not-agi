\documentclass[12pt]{article}
\usepackage[utf8]{inputenc}
\usepackage{amsmath,amssymb}
\usepackage{geometry}
\usepackage{setspace}
\usepackage{hyperref}
\geometry{margin=1in}
\onehalfspacing

\title{A Friendly Introduction to Logic and the Number Line:\\
From ``Good/Bad'' Labeling to Grains of Rice}
\author{Brian Searls\\\href{https://github.com/briansrls}{https://github.com/briansrls}}
\date{\today}

\begin{document}

\maketitle

\begin{abstract}
This paper offers an approachable explanation of what logic is and how
it underpins our idea of the number line. We treat logic as a system
for labeling statements as ``good'' (true, acceptable) or ``bad''
(false, unacceptable). We then connect this view to how we think about
numbers: from counting identical items (natural numbers), through
partial truths (decimals or fractions), and into the realm of debts
of truth (negative numbers). Grains of rice serve as a running analogy
to keep things down-to-earth.
\end{abstract}

\tableofcontents

\section{Introduction}
\label{sec:intro}

\subsection{Why Talk About Logic and Numbers Together?}
Logic is often described as the study of how we decide if a statement is
true or false. Meanwhile, numbers help us describe how many grains of rice
we have, how much water is in a cup, or how far we are from a certain point.
At first glance, these sound like separate topics. However, many philosophers
and mathematicians have argued that \emph{numbers grow out of logic}:
\begin{itemize}
  \item Counting objects is like piling up a series of ``true'' statements:
  ``This is one grain, here's another, now we have two,'' and so on.
  \item Handling fractions involves \emph{partial} truths, such as deciding
  that two halves really do make a whole (by agreement).
  \item Negative numbers let us speak of a ``lack'' or ``debts'' of something,
  as though we owe grains of rice to someone else and must keep track of that
  deficit.
\end{itemize}

In this paper, we'll give an informal, intuitive account of how logic and
numbers are connected, using a few examples that (we hope) are easy to visualize.

\section{Logic as Labeling Statements: ``Good'' vs. ``Bad''}
\label{sec:logic-good-bad}

\subsection{Binary Truth Values}
Classical logic usually works with two truth values:
\[
\textbf{True} \quad (\text{T}) \quad \leftrightarrow \quad 
\text{``Good/Acceptable''}
\]
\[
\textbf{False} \quad (\text{F}) \quad \leftrightarrow \quad 
\text{``Bad/Unacceptable''}
\]
A statement labeled \emph{true} is one we believe does not contradict any
other statement we've accepted as true. By contrast, if a statement is
labeled \emph{false}, we discard it as inconsistent with our accepted
facts or rules.

\subsection{Consistency and the Law of Noncontradiction}
The main rule that keeps logic consistent is:
\[
\text{``Do not accept a statement as both true and false 
   at the same time.''}
\]
This principle, called the \textbf{Law of Noncontradiction}, helps us avoid
the problem that anything could be ``proved'' once a contradiction is
accepted. If logic is our labeling system, then once a statement is labeled
``true,'' we don't want it also labeled ``false'' within the same system.

\subsection{Changing Perspectives, Same Underlying Discipline}
Even though different people might \emph{want} different statements to be
labeled ``true'' or ``false,'' the logical framework stays the same:
we only require that each person or group be \emph{consistent} in their
labeling. For example, one chef may say, ``These 100 grains of rice are
all high-quality (true), and the rest are bad (false).'' Another chef
may choose a different threshold for ``good'' quality. As long as neither
chef ends up calling the same grain both good and bad, they respect
the Law of Noncontradiction.

\section{Numbers: A Logical View with Grain-of-Rice Examples}
\label{sec:numbers}

\subsection{Natural Numbers: Counting Grain by Grain}
\subsubsection{``Fungible'' Truths}
When we say we have $1$ grain of rice, we are basically labeling:
\[
\text{``Here is 1 grain of rice.''} \quad (\text{true})
\]
If we pick up a second grain, we have:
\[
\text{``Here is another grain of rice.''} \quad (\text{true})
\]
From a logical perspective, the \emph{idea} of counting is to combine
multiple distinct but \emph{equally true} statements:
\[
1 + 1 + 1 + \dots
\]
We ignore that the grains might differ in shape or color; for counting
purposes, we treat them as \emph{fungible} (interchangeable) truths.
That yields the natural numbers $\{1, 2, 3, \ldots\}$: 
``1 grain, 2 grains, 3 grains, and so on.''

\subsubsection{Zero: ``Nothing in the Bag''}
When a bag is empty, we say there are $\textbf{0}$ grains inside. Logically,
this corresponds to saying: \emph{``There are no true statements of the
form `Here is a grain' inside this bag.''}

\subsection{Decimals and Fractions: Partial Truths}
\subsubsection{Half a Grain of Rice}
Imagine you cut a single grain into two (approximately equal) pieces. Now
you might label each piece as ``$0.5$ of a grain.'' In everyday usage,
two such halves make ``a full grain.'' Thus, a \emph{decimal} (or fraction)
expresses a partial \emph{truth}:
\[
0.5 + 0.5 = 1.0
\]
We \emph{agree} (by convention) that these halves \emph{add up} to one
whole. Of course, physically, the halves won't always be perfect, but
this is where a shared \emph{approximation} or \emph{agreement} underlies
our math. Logically, we decide that $0.5$ stands for ``half the usual
mass/volume of a grain.''

\subsection{Negative Numbers: Debts or ``Excess Falsehoods''}

\subsubsection{Owing Grains of Rice}
Now, imagine you borrowed 3 grains from a friend because you ran out. You
eat them, so you no longer have any grains in your possession. But logically,
you still \emph{owe} 3 grains. If you see your current count of grains as
``physical plus owed,'' you can be said to have \emph{negative 3 grains},
or $-3$. This means:
\[
\text{``I have a deficit of 3 grains relative to owning 0 grains.''}
\]
In other words, once we pass below zero, we are in the ``negative'' region,
representing debt or a shortfall. We can think of it as having more
``false'' (unfulfilled) statements than true ones:
\begin{itemize}
  \item A statement like ``I have the 3 grains to give back now'' is false.
  \item We track these unfulfilled truths as negatives.
\end{itemize}
When we finally acquire 3 new grains, we can pay off the debt and return
to zero.

\subsubsection{Combining Negatives and Positives}
If we owe 2 grains ($-2$) but then get 5 grains from somewhere else, we
add:
\[
-2 + 5 = 3.
\]
\emph{Logically}, we first use 2 of those grains to cancel our debt, and
then we have 3 grains left. Thus, negative numbers let us keep track of
things that we \emph{lack}, and they obey consistent rules of adding,
subtracting, and so forth.

\section{Constructing the Number Line via Logic}
\label{sec:num-line}

Putting it all together, we can see the \textbf{number line} from negative
to positive as a map of \emph{states} of truth about grains of rice (or
any item):
\begin{itemize}
  \item \textbf{Negative region} ($-1, -2, -3, \dots$) captures \emph{owing} or
  \emph{lacking} certain items.
  \item \textbf{Zero} is having no items, no surplus or deficit.
  \item \textbf{Positive region} ($1, 2, 3, \dots$) indicates how many items
  we currently hold.
  \item \textbf{Fractions/Decimals} fill in the partial steps along the way,
  e.g.\ $\frac{1}{2} = 0.5$ means we have half a unit of something.
\end{itemize}
All of these can be built from the basic idea of labeling statements in a
consistent way: how many full units do we have (positive), how many do
we owe (negative), and how do we split units into fractions (decimals).

\section{Tying Logic and Numbers Together}

\subsection{Logic: A Master Labeling System}
Logic itself remains the ``master'' labeling system. We have the rules of
``true'' and ``false'' and ensure we don't accidentally assign both to
the same statement. Whether we talk about \emph{owning}, \emph{owing},
or \emph{splitting} grains of rice, these moves rest on consistent
\emph{logical} rules that define what it means to add or subtract,
to have half, or to be in debt.

\subsection{Why This View Makes Sense}
\begin{enumerate}
  \item \textbf{Common-Sense Counting}: By ignoring physical differences in
  grains, we are free to just count how many. That is a \emph{logical
  simplification}.
  \item \textbf{Fractions as Agreement}: Two halves is one whole by
  \emph{convention}---that is a logical agreement among people, rather than
  an inevitable truth of the universe.
  \item \textbf{Negatives as Debts of Truth}: If you owe 3 grains, you are
  ``under zero'' by 3. This matches everyday arithmetic: your negative
  figure goes back to zero once you repay the grains.
\end{enumerate}

\section{Conclusion}
By viewing logic as a way to label statements as good or bad, we open
the door to understanding how \emph{numbers} emerge. In practice:
\begin{itemize}
  \item \textbf{Natural numbers} capture how many items we have---they
  come from repeatedly affirming ``Yes, I have a grain.''
  \item \textbf{Decimals/fractions} are partial truths---``I have 0.5
  grains'' becomes acceptable once we agree on how halves work.
  \item \textbf{Negative numbers} represent debt or deficits---as though
  we owe so many grains that we are below zero until we repay them.
\end{itemize}
All of these ideas rely on the same foundational logic that prevents
contradictions and keeps our labeling consistent. Whether we talk about
owning 10 grains, owing 3, or splitting 1 grain into halves, logic
remains the unifying system that keeps track of what is ``true''
(what we actually have, owe, or measure) and what is ``false''
(stating we have or owe something contradictory).

\begin{thebibliography}{9}

\bibitem{Boole}
G. Boole.
\textit{An Investigation of the Laws of Thought}.
Walton and Maberly, 1854.

\bibitem{Frege}
G. Frege.
\textit{The Foundations of Arithmetic}.
1884. (Various English translations available.)

\bibitem{RussellWhitehead}
B. Russell and A. N. Whitehead.
\textit{Principia Mathematica}.
Cambridge University Press, 1910--1913.

\bibitem{Godel}
K. G\"odel.
\textit{On Formally Undecidable Propositions of Principia Mathematica and Related Systems I}.
1931. (English translation available in various collections.)

\end{thebibliography}

\end{document}
